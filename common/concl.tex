%% Согласно ГОСТ Р 7.0.11-2011:
%% 5.3.3 В заключении диссертации излагают итоги выполненного исследования, рекомендации, перспективы дальнейшей разработки темы.
%% 9.2.3 В заключении автореферата диссертации излагают итоги данного исследования, рекомендации и перспективы дальнейшей разработки темы.
\begin{enumerate}
  \item Предложенный вариационный метод с радиальными базисными функциями обладает точностью, сравнимой с методом сферических гармоник, при этом являясь более экономичным. Использование диагонального углового предобуславливания позволяет значительно ускорить вычислительный алгоритм при использовании быстрых методов решения скалярных эллиптических уравнений типа квазидиффузии.
  \item Алгоритм упорядочения неструкрурированной сетки, примененный в маршевом методе коротких характеристик может быть использован в других численных методах, предназначенных для решения стационарных гиперболических задач. Хотя сам алгоритм упорядочения существенно последователен, результатом его работы является ярусно-параллельная форма графа зависимостей вычислительного алгоритма, которую можно использовать для распараллеливания процесса решения задачи.
  \item Предложен распределенный метод длинных характеристик, пригодоный для решения задачи переноса излучения на как на многопроцессорных вычислительных системах с распределенной памятью, так и на кластерах с графическими ускорителями.
  \item Разработанные вычислительные алгоритмы реализованы в программном комплексе. Использование неструктурированных тетраэдральных сеток позволяет применять комплекс в качестве универсального расчетного модуля в существующих программных гидродинамических пакетах.
  \item С помощью распределенного метода длинных характеристик на основании приближенной локально термодинамически равновесной модели поглощения численно построен профиль линии H-$\alpha$ в излучении звезды типа Т Тельца по результатам трехмерного МГД моделирования.
\end{enumerate}
