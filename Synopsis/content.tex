\subsection*{Общая характеристика работы}

\newcommand{\actuality}{\underline{\textbf{Актуальность темы.}}}
\newcommand{\aim}{\underline{\textbf{Целью}}}
\newcommand{\tasks}{\underline{\textbf{задачи}}}
\newcommand{\defpositions}{\underline{\textbf{Основные положения, выносимые на~защиту:}}}
\newcommand{\novelty}{\underline{\textbf{Научная новизна:}}}
\newcommand{\influence}{\underline{\textbf{Практическая значимость}}}
\newcommand{\reliability}{\underline{\textbf{Достоверность}}}
\newcommand{\probation}{\underline{\textbf{Апробация работы.}}}
\newcommand{\contribution}{\underline{\textbf{Личный вклад.}}}
\newcommand{\publications}{\underline{\textbf{Публикации.}}}

{\actuality}
Обзор, введение в тему, обозначение места данной работы в мировых исследованиях и~т.\:п.

 \aim\ данной работы является \ldots

Для~достижения поставленной цели необходимо было решить следующие {\tasks}:
\begin{enumerate}
  \item Исследовать, разработать, вычислить и~т.\:д. и~т.\:п.
  \item Исследовать, разработать, вычислить и~т.\:д. и~т.\:п.
  \item Исследовать, разработать, вычислить и~т.\:д. и~т.\:п.
  \item Исследовать, разработать, вычислить и~т.\:д. и~т.\:п.
\end{enumerate}

\defpositions
\begin{enumerate}
  \item Первое положение
  \item Второе положение
  \item Третье положение
  \item Четвертое положение
\end{enumerate}

\novelty
\begin{enumerate}
  \item Впервые \ldots
  \item Впервые \ldots
  \item Было выполнено оригинальное исследование \ldots
\end{enumerate}

\influence\ \ldots

\reliability\ полученных результатов обеспечивается \ldots \ Результаты находятся в соответствии с результатами, полученными другими авторами.

\probation\
Основные результаты работы докладывались~на:
перечисление основных конференций, симпозиумов и~т.\:п.

\contribution\ Автор принимал активное участие \ldots

\publications\ Основные результаты по теме диссертации изложены в 8 печатных изданиях~\cite{skalko2014, tsybulin2015a, tsybulin2015b},
2 из которых изданы в журналах, рекомендованных ВАК~\cite{skalko2014,tsybulin2015a}, 
5 --- в тезисах докладов~\cite{miptconf53,miptconf54,miptconf55,miptconf56,miptconf57}.
 % Характеристика работы по структуре во введении и в автореферате не отличается (ГОСТ Р 7.0.11, пункты 5.3.1 и 9.2.1), потому её загружаем из одного и того же внешнего файла, предварительно задав форму выделения некоторым параметрам

%Диссертационная работа была выполнена при поддержке грантов ...

%\underline{\textbf{Объем и структура работы.}} Диссертация состоит из~введения, четырех глав, заключения и~приложения. Полный объем диссертации \textbf{ХХХ}~страниц текста с~\textbf{ХХ}~рисунками и~5~таблицами. Список литературы содержит \textbf{ХХX}~наименование.

%\newpage
\subsection*{Содержание работы}
Во \underline{\textbf{введении}} обосновывается актуальность исследований, проводимых в рамках данной диссертационной работы, приводится обзор научной литературы по изучаемой проблеме, формулируется цель, ставятся задачи работы, сформулированы научная новизна и практическая значимость представляемой работы.

\underline{\textbf{Первая глава}} посвящена ...

 картинку можно добавить так:
\begin{figure}[ht] 
  \center
  \includegraphics [scale=0.27] {latex}
  \caption{Подпись к картинке.} 
  \label{img:latex}
\end{figure}

Формулы в строку без номера добавляются так:
\[ 
  \lambda_{T_s} = K_x\frac{d{x}}{d{T_s}}, \qquad
  \lambda_{q_s} = K_x\frac{d{x}}{d{q_s}},
\]

\underline{\textbf{Вторая глава}} посвящена исследованию 

\underline{\textbf{Третья глава}} посвящена исследованию 

В \underline{\textbf{четвертой главе}} приведено описание 

В \underline{\textbf{заключении}} приведены основные результаты работы, которые заключаются в следующем:
%% Согласно ГОСТ Р 7.0.11-2011:
%% 5.3.3 В заключении диссертации излагают итоги выполненного исследования, рекомендации, перспективы дальнейшей разработки темы.
%% 9.2.3 В заключении автореферата диссертации излагают итоги данного исследования, рекомендации и перспективы дальнейшей разработки темы.
\begin{enumerate}
  \item Для решения уравнения переноса излучения разработан вариационный метод с радиальными базисными функциями, который обладает точностью, сравнимой с методом сферических гармоник, но при этом более экономичный. Предложенное блочно-диагональное предобуславливание позволяет значительно ускорить вычислительный алгоритм. Построены оптимальные квадратурные формулы для полусферы, инвариантные относительно группы вращений.
  \item Разработан маршевый метод коротких характеристик. Построены варианты метода первого и второго порядка аппроксимации. Получено условие расположения узлов, выполнение которого необходимо для устойчивости метода второго порядка. Для монотонизации схемы второго порядка применен ограничитель значения интенсивности в дополнительных узлах.
  \item Для маршевого метода построены алгоритмы упорядочения неструктурированных сеток. Дополнительным результатом работы алгоритмов упорядочения является ярусно-параллельная форма графа зависимостей вычислительного метода, которую можно использовать для распараллеливания процесса решения задачи. 
  \item Разработана версия метода длинных характеристик, адаптированная для распределенной реализации на многопроцессорных системах и на кластерах с графическими ускорителями. Исследованы ускорение и эффективность реализаций в зависимости от числа используемых вычислительных узлов и графических ускорителей.
  \item Разработанные вычислительные алгоритмы реализованы в программном комплексе. В рамках модели локального термодинамического равновесия вычислен коэффициент поглощения частично ионизованной плазмы. Для задачи моделирования спектра излучения звезды типа Т Тельца построен спектральный профиль линии H-$\alpha$ в зависимости от ориентации плоскости аккреционного диска.
\end{enumerate}


\lipsum[1-70]

%\newpage
\renewcommand{\refname}{\large Публикации автора по теме диссертации}
\nocite{*}
\insertbiblioauthor                          % Подключаем Bib-базы
%\insertbibliofull
