\subsection*{Общая характеристика работы}


\newcommand{\contribution}{{\textbf{Личный вклад автора в публикации с соавторами.}}}

\newcommand{\actuality}{{\textbf{Актуальность и степень разработанности темы.}}}
\newcommand{\aim}{{\textbf{Целью}}}
\newcommand{\tasks}{{\textbf{задачи}}}
\newcommand{\novelty}{{\textbf{Научная новизна:}}}
\newcommand{\influence}{{\textbf{Теоретическая и практическая значимость}}}
\newcommand{\defpositions}{{\textbf{Основные положения, выносимые на~защиту}}}
\newcommand{\probation}{{\textbf{Степень достоверности и апробация работы.}}}
\newcommand{\methodology}{{\textbf{Методология и методы исследования.}}}

{\actuality} Перенос излучения является существенным процессом в задачах динамики высокотемпературной плазмы. Сильноточные разряды в установках на основе Z-пинчей являются мощными источниками рентгеновского излучения, которое может быть использовано, в частности, для экспериментального исследования процессов, протекающих в плазме с высокой плотностью и энергией. 
В астрофизике актуальной задачей является моделирование различных объектов, например, аккреационных дисков молодых звезд, квазаров, релятивистских струй и т. п.
Для этих задач судить об адекватности математической модели можно сравнивая наблюдаемый спектр объекта со спектром, получаемым в численных расчетах.
Перенос излучения важен и в метеорологии, так как он играет ключевую роль в атмосферном теплообмене и, как следствие, в формировании климата планеты.

Для задачи переноса излучения разработаны приближенные методы, учитывающие ту или иную симметрию (плоская, цилиндрическая или сферическая симметрия задачи), различные свойства коэффициента поглощения (приближения серой материи, оптически тонкого и оптически толстого слоя) и приближенные способы описания угловой зависимости излучения (приближение <<вперед-назад>>, диффузионное приближение). Существенного упрощения можно достичь при условии локализации источников излучения, например, на границе расчетной области.

В ряде процессов, например, сильноточных разрядах в установках на основе Z-пинчей или динамике околозвездного вещества, допустимо использование приближенных методов расчета излучения. Однако, хотя приближенные методы адекватны в гидродинамическом моделировании, их результатов недостаточно для сравнения с имеющимися экспериментальными данными. В этом случае, как правило, применяют постобработку результатов гидродинамического моделирования с использованием более точного метода решения задачи переноса излучения.

Однако, с развитием вычислительной техники, в особенности, использованием графических ускорителей, становится возможным решать задачу переноса излучения в полноценной трехмерной постановке с достаточно подробным описанием частотной и угловой зависимости решения. Задача переноса излучения обладает существенным запасом параллелизма, поскольку решения вдоль различных направлений и на различных частотах не зависят друг от друга. Привлекательным является также создание универсального программного блока, пригодного для работы совместно с различными существующими гидродинамическими программными комплексами. При этом существенно ограничивается свобода выбора таких параметров, как тип расчетной сетки, способ ее декомпозиции на вычислительные подобласти.

 \aim\ данной работы являются:
\begin{enumerate}
  \item Построение и исследование численных методов решения уравнения переноса.
  \item Реализация полученных методов с использованием графических ускорителей.
  \item Моделирование линейчатого спектра излучения звезды типа Т Тельца при наличии конического ветра. 
\end{enumerate}

Для~достижения поставленных целей были решены следующие {\tasks}:
\begin{enumerate}
  \item На основании вариационного принципа Владимирова построен численный метод решения уравнения переноса излучения для произвольного базиса угловых функций.
  \item Построена квадратурная формула для полусферы и разработан метод численного построения квадратурных формул продолжением по параметру.
  \item Изучены вопросы сходимости метода для базиса из сферических функций и базиса из радиальных функций.
  \item Создана параллельная реализация метода, использующая графический ускоритель.
  
  \item Построен маршевый алгоритм решения уравнения переноса излучения для неструктурированных тетраэдральных сеток методом коротких характеристик.
  \item Построен маршевый метод второго порядка и предложен способ монотонизации решения.
  \item Показана корректность данного алгоритма в случае использования тетраэдральной сетки, удовлетворяющей условию Делоне.
  \item Предложен алгоритм упорядочения для сеток, не удовлетворяющих условию Делоне. Предложено приложение алгоритма упорядочения для ярусно-параллельной формы графа зависимостей маршевого метода.
  
  \item Построен распределенный метод для решения уравнения переноса излучения, использующий длинные характеристики, ограниченные расчетной подобластью.
  \item Выведено трассировочное соотношение и доказана лемма об устойчивой трассировке.
  \item Метод реализован в многопроцессорном MPI варианте, а также в варианте, использующем кластер из графических ускорителей.
  \item Изучены вопросы ускорения и эффективности параллельной реализации.
  
  \item Построена упрощенная локально-термодинамически равновесная математическая модель поглощения излучения околозвездным веществом, учитывающая доплеровский сдвиг частоты поглощения и различную заселенность уровней атома водорода.
  \item По существующим результатам гидродинамического моделирования звезды типа Т Тельца проведен расчет спектра излучения в линии $\text{H-}\alpha$.
\end{enumerate}

\novelty
\begin{enumerate}
  \item Впервые был предложен вариационный метод для решения самосопряженного уравнения переноса излучения с базисом из радиальных угловых функций.
  \item Впервые был предложен маршевый алгоритм для решения уравнения переноса на неструктурированной тетраэдральной сетке.
  \item Предложена оригинальная распределенная многопроцессорная реализация метода длинных характеристик.
\end{enumerate}

\influence\ Предложены и исследованы новые методы решения уравнения переноса излучения. Изучены вопросы распараллеливания предложенных методов. Получены фактические оценки скорости сходимости по пространственным и угловым переменным.

Предложенные алгоритмы упорядочения неструктурированных третраэдральных сеток могут применяться для решения других стационарных гиперболических задач как конечно-разностными, так и конечно-объёмными численными методами.

Разработанные программы возможно использовать в существующих гидродинамических программных комплексах, в том числе реализованных для кластерных вычислительных систем.

Работа выполнялась при поддержке проекта Министерства образования и науки РФ №3.522.2014/К <<Исследование процессов, происходящих в веществе, и изменения его свойств при импульсном вводе энергии высокой плотности>>.

\methodology\ В основе численных методов лежат вариационный метод Ритца и различные сеточно-характеристические методы. Для решения систем линейных уравнений применяются как итерационный метод сопряженных градиентов, так и прямые методы разложения разреженной матрицы. 
В основе распределенного метода лежит принцип геометрической декомпозиции расчетной области.
Анализ численной сходимости проводился на задачах, имеющих аналитическое решение.

\defpositions\ соответствуют основным результатам диссертации, приведенным в заключении.

\probation\ Основные результаты по теме диссертации изложены в 8 печатных изданиях~\cite{skalko2014, tsybulin2015a, tsybulin2015b},
2 из которых изданы в журналах, рекомендованных ВАК~\cite{skalko2014,tsybulin2015a}, 
5 --- в тезисах докладов~\cite{miptconf53,miptconf54,miptconf55,miptconf56,miptconf57}.

Основные результаты работы докладывались и получили одобрение на следующих научных конференциях:
\begin{enumerate}
\item 53-й научной конференции МФТИ <<Современные проблемы фундаментальных и прикладных наук>>, Долгопрудный, 2010.
\item 54-й научной конференции МФТИ <<Проблемы фундаментальных и прикладных естественных и технических наук в современном информационном обществе>>, Долгопрудный, 2011.
\item 55-й научной конференции МФТИ <<Проблемы фундаментальных и прикладных естественных и технических наук в современном информационном обществе>>, Долгопрудный, 2012.
\item 56-й научной конференции МФТИ <<Актуальные проблемы фундаментальных и прикладных наук в современном информационном обществе>>, Долгопрудный, 2013.
\item 57-й научной конференции МФТИ <<Актуальные проблемы фундаментальных и прикладных наук в области физики>>, Долгопрудный, 2014.
\end{enumerate}

Основные результаты работы докладывались и получили одобрение на следующих семинарах:
\begin{enumerate}
\item Лаборатория математического моделирования нелинейных процессов в газовых средах, МФТИ, Москва, 2012.
\item Научная сессия VII школы по высокопроизводительным вычислениям, Университет Иннополис, Казань, 2015.
\item Лаборатория флюидодинамики и сейсмоакустики, МФТИ, Москва, 2015.
\item Институт автоматизации проектирования РАН, Москва, 2015.
\end{enumerate}
 % Характеристика работы по структуре во введении и в автореферате не отличается (ГОСТ Р 7.0.11, пункты 5.3.1 и 9.2.1), потому её загружаем из одного и того же внешнего файла, предварительно задав форму выделения некоторым параметрам

\textbf{Объём и структура работы.} Диссертация состоит из~введения, \todo{четырёх} глав, заключения и \todo{двух} приложений. Полный объём диссертации составляет \todo{X} страниц
с~\todo{Y} рисунками и \todo{Z} таблицами. Список литературы содержит \todo{W} наименований.

%\newpage
\subsection*{Содержание работы}
Во \underline{\textbf{введении}} обосновывается актуальность исследований, проводимых в рамках данной диссертационной работы, приводится обзор научной литературы по изучаемой проблеме, формулируется цель, ставятся задачи работы, сформулированы научная новизна и практическая значимость представляемой работы.

\underline{\textbf{Первая глава}} посвящена ...

 картинку можно добавить так:
\begin{figure}[ht] 
  \center
  \includegraphics [scale=0.27] {latex}
  \caption{Подпись к картинке.} 
  \label{img:latex}
\end{figure}

Формулы в строку без номера добавляются так:
\begin{gather}
  \upalpha \alpha \leq \upbeta \beta \neq \emptyset \uppi \pi
\end{gather}

\underline{\textbf{Вторая глава}} посвящена исследованию 

\underline{\textbf{Третья глава}} посвящена исследованию 

В \underline{\textbf{четвертой главе}} приведено описание 

В \underline{\textbf{заключении}} приведены основные результаты работы, которые заключаются в следующем:
%% Согласно ГОСТ Р 7.0.11-2011:
%% 5.3.3 В заключении диссертации излагают итоги выполненного исследования, рекомендации, перспективы дальнейшей разработки темы.
%% 9.2.3 В заключении автореферата диссертации излагают итоги данного исследования, рекомендации и перспективы дальнейшей разработки темы.
\begin{enumerate}
  \item Для решения уравнения переноса излучения разработан вариационный метод с радиальными базисными функциями, который обладает точностью, сравнимой с точностью метода сферических гармоник, но при этом является более экономичным. Экономичность была достигнута за счет использования
  предложенного блочно-диагонального предобуславливания метода решения системы линейных уравнений. 
  Построены оптимальные квадратурные формулы для полусферы, инвариантные относительно группы вращений.
  \item Разработан маршевый метод коротких характеристик. Построены варианты данного метода первого и второго порядка аппроксимации. Получено условие расположения узлов, выполнение которого необходимо для устойчивости метода второго порядка. Для монотонизации схемы второго порядка применен ограничитель значения интенсивности в дополнительных узлах.
  \item Для маршевого метода построены алгоритмы упорядочения неструктурированных сеток. Дополнительным результатом работы алгоритмов упорядочения является ярусно-параллельная форма графа зависимостей вычислительного метода, которую можно использовать для распараллеливания процесса решения задачи. 
  \item Разработана версия метода длинных характеристик, адаптированная для распределенной реализации на многопроцессорных системах и на кластерах с графическими ускорителями. Исследованы ускорение и эффективность реализаций метода в зависимости от числа используемых вычислительных узлов и графических ускорителей.
  \item Разработанные вычислительные алгоритмы реализованы в виде программного комплекса. В рамках модели локального термодинамического равновесия вычислен коэффициент поглощения частично ионизованной плазмы. Для задачи моделирования спектра излучения звезды типа Т Тельца построен спектральный профиль линии H-$\alpha$ в зависимости от ориентации плоскости аккреционного диска.
\end{enumerate}


%\lipsum[1-70]

%\newpage
\renewcommand{\refname}{\large Публикации автора по теме диссертации}
\nocite{*}
\insertbiblioauthor                          % Подключаем Bib-базы
%\insertbibliofull

\subsection*{\contribution} 
В работе \cite{skalko2014} автором были предложены и реализованы два варианта численного алгоритма для решения уравнения переноса, Скалько Ю.И. был предложен алгоритм упорядочения тетраэдров по проекциям их центров, а Карасевым Р.Н. и Акопяном А.В. была доказана его корректность для триангуляций Делоне. Мендель М.А. выполнил расчеты, изучил порядок сходимости по пространственной переменной и обнаружил эффект насыщения по угловым направлениям.

В работе \cite{tsybulin2015a} автору принадлежит идея распределенного метода длинных характеристик и его гибридная MPI-CUDA реализация. Скалько Ю.И. было предложено улучшение балансировки нагрузки за счет решения нескольких линейных систем одновременно на разных вычислительных узлах. Павловой Е.С. была реализована модификация, позволяющая решать уравнение переноса излучения в многогрупповом приближении.

В работах \cite{tsybulin2015b, miptconf55, miptconf56, miptconf57, miptconf54} автором построены численные методы для решения уравнения переноса, выполнена программная реализация на графических ускорителях методов численного решения систем линейных и линейных обыкновенных уравнений, поставлена задача построения квадратурной формулы для полусферы и предложен метод ее численного решения. 