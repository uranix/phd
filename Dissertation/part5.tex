\chapter{Результаты}

Приведены результаты исследования построенных методов.

\section{Исследование сходимости метода, основанного на вариационном принципе}

Методы имеют в областях гладкости второй порядок сходимости.

\subsection{Сходимость метода сферических гармоник}

Метод сходится со скоростью $\varepsilon_{L_2} \sim K^{-0.29}$.

\subsection{Сходимость метода с радиальными базисными функциями}

Метод сходится аналогично методу сферических гармоник.

\subsection{Сходимость метода при использовании предобуславливателя}

Метод сферических гармоник сходится за $30 \dividesymbol 100$ итераций предобуславливателя.

\section{Исследование маршевого метода коротких характеристик}

\subsection{Сравнение численной диффузии луча в методах первого и второго порядка}

\subsection{Сходимость методов первого и второго порядка}

Метод первого порядка сходится линейно, а второго --- квадратично.

\section{Исследование распределенного метода длинных характеристик}

\subsection{Сравнение решений для различного числа подобластей}

Для нескольких подобластей наблюдается незначительная диффузия луча, проходящего через границу подобластей.

\subsection{Ускорение и эффективность параллельной реализации}

Версия, использующая графические ускорители в $3.5$ раз быстрее версии для центрального процессора.

\section{Расчет спектра излучения линии H-$\alpha$ звезды типа Т Тельца}

\section{Выводы}
