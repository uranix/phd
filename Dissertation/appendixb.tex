\chapter{Сферические функции}
\label{chap:spher}

\section{Сферические функции, используемые в работе}

В работе были использованы действительные сферические функции, образующие ортонормированную систему по отношению
к скалярному произведению $(\varphi, \psi) = \frac{1}{4\pi} \int \varphi \psi d\Omega$. Они имеют следующее представление
\begin{align*}
Y_{l,m}(\theta, \varphi) &= C_{l,m}P_l^{m}(\cos \theta)\times
\begin{cases}
\sin |m| \varphi,&\, m<0\\
\frac{1}{\sqrt{2}},&\, m=0\\
\cos m \varphi,&\, m>0\\
\end{cases}
\label{eq:sphfunc}\\
C_{l,m} &= (-1)^{\frac{|m|+m}{2}}\sqrt{4l+2}\sqrt{\frac{(l-m)!}{(l+m)!}},
\end{align*}
где $P^m_l(\mu)$ --- присоединенный многочлен Лежандра степени $l$ и порядка $m$.

Сферические функции могут быть выражены как однородные многочлены от компонент $\vec \Omega$. Представление нескольких первых 
функций приведено ниже:
\begin{align*}
Y_{0,0} &= 1\\
Y_{1,-1} = \sqrt{3}\Omega_y,\quad
Y_{1,0} &= \sqrt{3}\Omega_z,\quad
Y_{1,1} = \sqrt{3}\Omega_x\\
Y_{2,-2} = \sqrt{15}\Omega_x\Omega_y,\quad
Y_{2,-1} &= \sqrt{15}\Omega_y\Omega_z, \quad
Y_{2,1} = \sqrt{15}\Omega_x\Omega_z\\
Y_{2,0} = \sqrt{5}\Omega_z^{2}&-\frac{\sqrt{5}}{2}\Omega_y^{2}-\frac{\sqrt{5}}{2}\Omega_x^{2}\\
Y_{2,2} = \frac{\sqrt{15}}{2}\Omega_x^{2}&-\frac{\sqrt{15}}{2}\Omega_y^{2}
\end{align*}

\section{Правила отбора для интеграла от трех действительных сферических функций}

Рассмотрим следующий интеграл
\[
\begin{bmatrix}
l_1 & l_2 & l_3\\
m_1 & m_2 & m_3
\end{bmatrix} \equiv
\int\limits_{4\pi}
Y_{l_1,m_1}(\vec \Omega)
Y_{l_2,m_2}(\vec \Omega)
Y_{l_3,m_3}(\vec \Omega)
d\Omega.
\]
Без ограничений общности, предположим, что $l_1 \leq l_2 \leq l_3$.
Найдем необходимые условия, при которых он отличен от нуля. Учтем, что сферические функции допускают представление
\[
Y_{l,m}(\vec \Omega) = C_{l,m} P_l^{m}(\cos \theta) \Phi_m(\varphi),
\]
а, следовательно, рассматриваемый интеграл может быть записан как
\begin{multline*}
\begin{bmatrix}
l_1 & l_2 & l_3\\
m_1 & m_2 & m_3
\end{bmatrix} =
C_{l_1,m_1}
C_{l_2,m_2}
C_{l_3,m_3} \times \\ \times
\int\limits_{-1}^1 
P_{l_1}^{m_1}(\mu)
P_{l_2}^{m_2}(\mu)
P_{l_3}^{m_3}(\mu)
d\mu
\int\limits_{0}^{2\pi}
\Phi_{m_1}(\varphi)
\Phi_{m_2}(\varphi)
\Phi_{m_3}(\varphi)
d\varphi
\end{multline*}
Для интеграла
\[
\int\limits_{-1}^1 
P_{l_1}^{m_1}(\mu)
P_{l_2}^{m_2}(\mu)
P_{l_3}^{m_3}(\mu)
d\mu
\]
известны \cite{Gaunt1929} правила отбора, которые заключаются в том, что интеграл не равен нулю только если выполнены следующие предположения:
\begin{itemize}
\item сумма степеней $l_1 + l_2 + l_3$ четна;
\item степени удовлетворяют неравенству треугольника $l_3 < l_1 + l_2$.
\end{itemize}

Для интеграла 
\[
\int\limits_{0}^{2\pi}
\Phi_{m_1}(\varphi)
\Phi_{m_2}(\varphi)
\Phi_{m_3}(\varphi)
d\varphi
\]
рассмотрим возможные случаи.
\paragraph{Случай $m_1 = 0$.} Поскольку $\Phi_0(\varphi) = \frac{1}{\sqrt{2}}$, интеграл сводится к 
\[
\int\limits_{0}^{2\pi}
\Phi_{m_2}(\varphi)
\Phi_{m_3}(\varphi)
d\varphi,
\]
который отличен от нуля только при $m_2 = m_3$.
\paragraph{Случай $m_1 \leq m_2 \leq m_3 < 0$.} Интеграл представим в виде
\begin{multline*}
-\int\limits_{0}^{2\pi}
\sin m_1 \varphi
\sin m_2 \varphi
\sin m_3 \varphi
d\varphi = \\ 
= 
\frac{1}{2}\int\limits_{0}^{2\pi}
\sin m_1 \varphi
\cos (m_2 + m_3) \varphi
d\varphi 
-
\frac{1}{2}\int\limits_{0}^{2\pi}
\sin m_1 \varphi
\cos (m_3 - m_2) \varphi
d\varphi.
\end{multline*}
Оба интеграла равны нулю. 

\paragraph{Случай $m_1 \leq m_2 < 0 < m_3$.} Интеграл представим в виде
\begin{multline*}
\int\limits_{0}^{2\pi}
\sin m_1 \varphi
\sin m_2 \varphi
\cos m_3 \varphi
d\varphi = \\ 
= 
\frac{1}{2}\int\limits_{0}^{2\pi}
\cos (m_2 - m_1) \varphi
\cos m_3 \varphi
d\varphi 
+
\frac{1}{2}\int\limits_{0}^{2\pi}
\cos (m_1 + m_2) \varphi
\cos m_3 \varphi
d\varphi.
\end{multline*}
Интеграл отличен от нуля если $m_3 = m_2 - m_1$ либо $m_3 = -m_1 - m_2$.

\paragraph{Случай $m_1 < 0 < m_2 \leq m_3$.} Интеграл представим в виде
\begin{multline*}
-\int\limits_{0}^{2\pi}
\sin m_1 \varphi
\cos m_2 \varphi
\cos m_3 \varphi
d\varphi = \\ 
= 
-\frac{1}{2}\int\limits_{0}^{2\pi}
\sin m_1 \varphi
\cos (m_2 + m_3) \varphi
d\varphi 
-
\frac{1}{2}\int\limits_{0}^{2\pi}
\sin m_1 \varphi
\cos (m_2 - m_3) \varphi
d\varphi.
\end{multline*}
Оба интеграла равны нулю. 

\paragraph{Случай $0 < m_1 \leq m_2 \leq m_3$.} Интеграл представим в виде
\begin{multline*}
\int\limits_{0}^{2\pi}
\cos m_1 \varphi
\cos m_2 \varphi
\cos m_3 \varphi
d\varphi = \\ 
= 
\frac{1}{2}\int\limits_{0}^{2\pi}
\cos m_1 \varphi
\cos (m_2 + m_3) \varphi
d\varphi 
+
\frac{1}{2}\int\limits_{0}^{2\pi}
\cos m_1 \varphi
\cos (m_2 - m_3) \varphi
d\varphi.
\end{multline*}
Интеграл отличен от нуля только если $m_1 + m_2 = m_3$ (случай $m_1 = m_2 + m_3$ невозможен, т.к. $m_3 \geq m_1, m_2 > 0$).

Подытожим, интеграл 
\[
\int\limits_{0}^{2\pi} \Phi_{m_1}(\varphi)\Phi_{m_2}(\varphi)\Phi_{m_3}(\varphi) d\varphi, \quad 
m_1 \leq m_2 \leq m_3,
\]
отличен от нуля только если
\begin{itemize}
\item одно из чисел $m_1, m_2, m_3$ равно нулю, а остальные равны между собой;
\item либо $m_1, m_2, m_3 > 0$ и $m_3 = m_1 + m_2$;
\item либо $m_1, m_2 < 0, m_3 > 0$, при этом либо $m_1 + m_2 + m_3 = 0$, либо $m_1 - m_2 + m_3 = 0$.
\end{itemize}
В совокупности с правилами отбора для степеней $l_1, l_2, l_3$
\begin{itemize}
\item сумма степеней $l_1 + l_2 + l_3$ четна;
\item степени удовлетворяют неравенству треугольника $l_3 < l_1 + l_2$,
\end{itemize}
эти требования составляют правила отбора для интеграла
\[
\begin{bmatrix}
l_1 & l_2 & l_3\\
m_1 & m_2 & m_3
\end{bmatrix} \equiv
\int\limits_{4\pi}
Y_{l_1,m_1}(\vec \Omega)
Y_{l_2,m_2}(\vec \Omega)
Y_{l_3,m_3}(\vec \Omega)
d\Omega.
\]

\section{Матрица
\texorpdfstring{$\mathscr{K}^{\alpha\beta}_{kk'}$}{K^ab\_kk'}
приближения \texorpdfstring{$n = 1$}{n = 1}}

Для иллюстрации разреженной структуры матрицы, приведен вариант при $n = 2$. Блоки соответствуют следующей нумерации сферических гармоник: 
$\varphi_{0,0}\quad \varphi_{2,-2}\quad \varphi_{2,-1}\quad \varphi_{2,0}\quad \varphi_{2,1}\quad \varphi_{2,2}$

\begin{small}
\begin{align*}
%&\langle \Omega_i \Omega_j \rangle_{2k\,m}^{2k'\,m'} = \\
%&
\left(
\begin{array}{@{}c@{}c@{}c||c@{}cc|cc@{}c|c@{}c@{}c|ccc|@{}c@{}c@{}c@{}}
\frac{1}{3}&0&0&
0&\frac{\sqrt{15}}{15}&0&
0&0&0&
-\frac{\sqrt{5}}{15}&0&0&
0&0&\frac{\sqrt{15}}{15}&
\frac{\sqrt{15}}{15}&0&0\\
0&\frac{1}{3}&0&
\frac{\sqrt{15}}{15}&0&0&
0&0&\frac{\sqrt{15}}{15}&
0&-\frac{\sqrt{5}}{15}&0&
0&0&0&
0&-\frac{\sqrt{15}}{15}&0\\
0&0&\frac{1}{3}&
0&0&0&
0&\frac{\sqrt{15}}{15}&0&
0&0&\frac{2\sqrt{5}}{15}&
\frac{\sqrt{15}}{15}&0&0&
0&0&0\\
\hline
\hline
0&\frac{\sqrt{15}}{15}&0&
\frac{3}{7}&0&0&
0&0&\frac{1}{7}&
0&-\frac{2\sqrt{3}}{21}&0&
0&0&0&
0&0&0\\
\frac{\sqrt{15}}{15}&0&0&
0&\frac{3}{7}&0&
0&0&0&
-\frac{2\sqrt{3}}{21}&0&0&
0&0&\frac{1}{7}&
0&0&0\\
0&0&0&
0&0&\frac{1}{7}&
\frac{1}{7}&0&0&
0&0&0&
0&\frac{1}{7}&0&
0&0&0\\
\hline
0&0&0&
0&0&\frac{1}{7}&
\frac{1}{7}&0&0&
0&0&0&
0&\frac{1}{7}&0&
0&0&0\\
0&0&\frac{\sqrt{15}}{15}&
0&0&0&
0&\frac{3}{7}&0&
0&0&\frac{\sqrt{3}}{21}&
\frac{1}{7}&0&0&
0&0&-\frac{1}{7}\\
0&\frac{\sqrt{15}}{15}&0&
\frac{1}{7}&0&0&
0&0&\frac{3}{7}&
0&\frac{\sqrt{3}}{21}&0&
0&0&0&
0&-\frac{1}{7}&0\\
\hline
-\frac{\sqrt{5}}{15}&0&0&
0&-\frac{2\sqrt{3}}{21}&0&
0&0&0&
\frac{5}{21}&0&0&
0&0&\frac{\sqrt{3}}{21}&
-\frac{2\sqrt{3}}{21}&0&0\\
0&-\frac{\sqrt{5}}{15}&0&
-\frac{2\sqrt{3}}{21}&0&0&
0&0&\frac{\sqrt{3}}{21}&
0&\frac{5}{21}&0&
0&0&0&
0&\frac{2\sqrt{3}}{21}&0\\
0&0&\frac{2\sqrt{5}}{15}&
0&0&0&
0&\frac{\sqrt{3}}{21}&0&
0&0&\frac{11}{21}&
\frac{\sqrt{3}}{21}&0&0&
0&0&0\\
\hline
0&0&\frac{\sqrt{15}}{15}&
0&0&0&
0&\frac{1}{7}&0&
0&0&\frac{\sqrt{3}}{21}&
\frac{3}{7}&0&0&
0&0&\frac{1}{7}\\
0&0&0&
0&0&\frac{1}{7}&
\frac{1}{7}&0&0&
0&0&0&
0&\frac{1}{7}&0&
0&0&0\\
\frac{\sqrt{15}}{15}&0&0&
0&\frac{1}{7}&0&
0&0&0&
\frac{\sqrt{3}}{21}&0&0&
0&0&\frac{3}{7}&
\frac{1}{7}&0&0\\
\hline
\frac{\sqrt{15}}{15}&0&0&
0&0&0&
0&0&0&
-\frac{2\sqrt{3}}{21}&0&0&
0&0&\frac{1}{7}&
\frac{3}{7}&0&0\\
0&-\frac{\sqrt{15}}{15}&0&
0&0&0&
0&0&-\frac{1}{7}&
0&\frac{2\sqrt{3}}{21}&0&
0&0&0&
0&\frac{3}{7}&0\\
0&0&0&
0&0&0&
0&-\frac{1}{7}&0&
0&0&0&
\frac{1}{7}&0&0&
0&0&\frac{1}{7}
\end{array}
\right)
\end{align*}
\end{small}