\chapter{Распределенный метод длинных характеристик}

Предлагается версия метода длинных характеристик, не требующая трассировать лучи через всю область, состоящую из множества подобластей.

\section{Функция Грина для задачи переноса излучения}

Интенсивность в точке $\vec r$ связана с интенсивностью в точке $\vec r_0 = \vec r - \vec \Omega s$ соотношением
\[
I(\vec r, \vec \Omega) = \alpha(0, s) I(\vec r_0, \vec \Omega)  + \beta(0, s)
\]

\section{Устойчивость трассировки луча}

Для придания трассировке устойчивости предлагается смещать входную точку в тетраэдре.

\section{Распределенный метод}

В распределенном методе трассировка происходит лишь в пределах вычислительной подобласти.

\section{Параллельная реализация вычислительного алгоритма}

Основной этап вычислений реализован с использованием графических ускорителей.