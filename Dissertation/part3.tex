\chapter{Маршевый метод коротких характеристик}

\section{Численные методы для стационарных гиперболических задач}

Метод коротких характеристик и разрывный метод Галеркина приводят к необходимости  упорядочивать элементы сетки для безытерационного метода.

\section{Метод второго порядка аппроксимации}

Для повышения порядка аппроксимации предлагается использовать дополнительные узлы на ребрах тетраэдров.

\subsection{Расположение узловых точек}

Если вершины тетраэдров не использовать в качестве узлов интерполяции, метод может потерять устойчивость.

\subsection{Монотонизация схемы}

Для монотонизации схемы предлагается использовать ограничитель $\operatorname{clamp}$.

\section{Алгоритмы упорядочения сеточных элементов}
\subsection{Алгоритм для триангуляций Делоне}

Для триангуляций Делоне упорядочивать тетраэдры можно по проекции их центра на направление излучения.

\subsection{Алгоритм для триангуляций общего вида}

Для триангуляций общего вида упорядочивать тетраэдры можно алгоритмом, основанным на поиске в ширину.