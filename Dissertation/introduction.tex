\chapter*{Введение}							% Заголовок
\addcontentsline{toc}{chapter}{Введение}	% Добавляем его в оглавление

{\actuality}
Обзор, введение в тему, обозначение места данной работы в мировых исследованиях и~т.\:п.

 \aim\ данной работы является \ldots

Для~достижения поставленной цели необходимо было решить следующие {\tasks}:
\begin{enumerate}
  \item Исследовать, разработать, вычислить и~т.\:д. и~т.\:п.
  \item Исследовать, разработать, вычислить и~т.\:д. и~т.\:п.
  \item Исследовать, разработать, вычислить и~т.\:д. и~т.\:п.
  \item Исследовать, разработать, вычислить и~т.\:д. и~т.\:п.
\end{enumerate}

\defpositions
\begin{enumerate}
  \item Первое положение
  \item Второе положение
  \item Третье положение
  \item Четвертое положение
\end{enumerate}

\novelty
\begin{enumerate}
  \item Впервые \ldots
  \item Впервые \ldots
  \item Было выполнено оригинальное исследование \ldots
\end{enumerate}

\influence\ \ldots

\reliability\ полученных результатов обеспечивается \ldots \ Результаты находятся в соответствии с результатами, полученными другими авторами.

\probation\
Основные результаты работы докладывались~на:
перечисление основных конференций, симпозиумов и~т.\:п.

\contribution\ Автор принимал активное участие \ldots

\publications\ Основные результаты по теме диссертации изложены в 8 печатных изданиях~\cite{skalko2014, tsybulin2015a, tsybulin2015b},
2 из которых изданы в журналах, рекомендованных ВАК~\cite{skalko2014,tsybulin2015a}, 
5 --- в тезисах докладов~\cite{miptconf53,miptconf54,miptconf55,miptconf56,miptconf57}.
 % Характеристика работы по структуре во введении и в автореферате не отличается (ГОСТ Р 7.0.11, пункты 5.3.1 и 9.2.1), потому её загружаем из одного и того же внешнего файла, предварительно задав форму выделения некоторым параметрам

%% регистрируем счётчики в системе totcounter
\regtotcounter{totalcount@figure}
\regtotcounter{totalcount@table}       % Если поставить в преамбуле то ошибка в числе таблиц
\regtotcounter{TotPages}               % Если поставить в преамбуле то ошибка в числе страниц

\textbf{Объём и структура работы.} Диссертация состоит из~введения, четырёх глав, заключения и~двух приложений.
%% на случай ошибок оставляю исходный кусок на месте, закомментированным
%Полный объём диссертации составляет  \ref*{TotPages}~страницу с~\totalfigures{}~рисунками и~\totaltables{}~таблицами. Список литературы содержит \total{citenum}~наименований.
%
Полный объём диссертации составляет \formbytotal{TotPages}{страниц}{у}{ы}{} 
с~\formbytotal{totalcount@figure}{рисунк}{ом}{ами}{ами}
и~\formbytotal{totalcount@table}{таблиц}{ей}{ами}{ами}. Список литературы содержит  
\formbytotal{citenum}{наименован}{ие}{ия}{ий}.

\section*{Обзор существующих методов решения уравнения переноса излучения}
\addcontentsline{toc}{section}{Обзор существующих методов решения уравнения переноса излучения}

\todo{...}
