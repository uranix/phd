\chapter*{Заключение}						% Заголовок
\addcontentsline{toc}{chapter}{Заключение}	% Добавляем его в оглавление

%% Согласно ГОСТ Р 7.0.11-2011:
%% 5.3.3 В заключении диссертации излагают итоги выполненного исследования, рекомендации, перспективы дальнейшей разработки темы.
%% 9.2.3 В заключении автореферата диссертации излагают итоги данного исследования, рекомендации и перспективы дальнейшей разработки темы.
%% Поэтому имеет смысл сделать эту часть общей и загрузить из одного файла в автореферат и в диссертацию:

Основные результаты работы:
%% Согласно ГОСТ Р 7.0.11-2011:
%% 5.3.3 В заключении диссертации излагают итоги выполненного исследования, рекомендации, перспективы дальнейшей разработки темы.
%% 9.2.3 В заключении автореферата диссертации излагают итоги данного исследования, рекомендации и перспективы дальнейшей разработки темы.
\begin{enumerate}
  \item Для решения уравнения переноса излучения разработан вариационный метод с радиальными базисными функциями, который обладает точностью, сравнимой с методом сферических гармоник, но при этом более экономичный. Предложенное блочно-диагональное предобуславливание позволяет значительно ускорить вычислительный алгоритм. Построены оптимальные квадратурные формулы для полусферы, инвариантные относительно группы вращений.
  \item Разработан маршевый метод коротких характеристик. Построены варианты метода первого и второго порядка аппроксимации. Получено условие расположения узлов, выполнение которого необходимо для устойчивости метода второго порядка. Для монотонизации схемы второго порядка применен ограничитель значения интенсивности в дополнительных узлах.
  \item Для маршевого метода построены алгоритмы упорядочения неструктурированных сеток. Дополнительным результатом работы алгоритмов упорядочения является ярусно-параллельная форма графа зависимостей вычислительного метода, которую можно использовать для распараллеливания процесса решения задачи. 
  \item Разработана версия метода длинных характеристик, адаптированная для распределенной реализации на многопроцессорных системах и на кластерах с графическими ускорителями. Исследованы ускорение и эффективность реализаций в зависимости от числа используемых вычислительных узлов и графических ускорителей.
  \item Разработанные вычислительные алгоритмы реализованы в программном комплексе. В рамках модели локального термодинамического равновесия вычислен коэффициент поглощения частично ионизованной плазмы. Для задачи моделирования спектра излучения звезды типа Т Тельца построен спектральный профиль линии H-$\alpha$ в зависимости от ориентации плоскости аккреционного диска.
\end{enumerate}


Дальнейшее развитие метода, основанного на вариационном принципе возможно в направлении ускорения работы за счет использования быстрых методов решения эллиптических задач типа квазидиффузии, а также развитие метода для более сложных граничных условий и коэффициента поглощения, зависящего от направления. Использование распределенных методов решения уравнения квазидиффузии позволит реализовать этот метод в распределенном варианте.

Сохранение физичности решения и выполнение принципа максимума являются трудными задачами в методе, основанном на вариационном принципе, и, по-видимому, требуют введения нелинейности в целевой минимизируемый функционал.

Перспективным видится объединение метода коротких характеристик и распределенного метода длинных характеристик. Метод коротких характеристик позволит достаточно быстро построить разностный оператор Грина, хотя в данном случае он уже не может быть представлен компактно из-за численной диффузии луча. Однако, можно выделить небольшое число (например, 20) узлов, где содержится существенная часть носителя оператора Грина. При этом идея составления линейной системы, включающей только граничные неизвестные, остается без изменения.
